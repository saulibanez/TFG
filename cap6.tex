\chapter{Conclusiones}
\label{cap:conclusiones}

Los objetivos de este trabajo de fin de grado han sido dos: por un lado, presentar a los robots submarinos, en particular, OpenROV y por otro lado, la integración de ROS, el framework de robótica más extendido y utilizado con la tecnología del ROV.

Se realizó el montaje y la configuración de cero, por lo que he podido aprender sobre la tecnología del ROV, además de explorar los paquetes necesarios de ROS para realizar la comunicacion entre ROS y OpenROV.

Tras finalizar el montaje y el sistema del OpenROV aprendí a solventar imprevistos que no se contemplaban tanto de hardware como de software. En este aspecto, la comunidad de OpenROV ha sido de inestimable ayuda en algunos momentos.

Se comprobó que el robot es estanco y puede navegar bajo la superficie marina, que todos los controles mencionados en los puntos anteriores funcionan correctamente (láser, cámara, luces, etc) y que es probable que aparezcan algunos de los problemas comunes (por ejemplo, el vaho).

Como OpenROV tiene software libre, se pudo instalar los paquetes necesarios para realizar la comunicación con ROS y se comprobó que existía la comunicación entre las dos plataformas. Al final, el OpenROV puede mejorar bastante si se siguen implementando nodos de ROS para su utlilización (por ejemplo, la cámara, usando el paquete sensors, la velocidad, utilizando geometry\_msgs, etc). 

Para finalizar, otro de los siguientes pasos que se puede realizar con este TFG, es intentar comprobar si la leyenda de Nessie es cierta.
