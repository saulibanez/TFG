\chapter{Conclusiones}
\label{cap:conclusiones}

Los objetivos de este trabajo de fin de grado han sido dos. Por un lado, presentar a los robots submarinos, en particular, OpenROV, y por otro lado, la integración de ROS, el framework de robótica más extendido y utilizado con la tecnología del ROV.

Al realizar todo el montaje y la configuración desde cero, he podido conocer amplios aspectos acerca de la tecnología del ROV, además de explorar los paquetes necesarios de ROS para realizar la comunicacion entre ROS y OpenROV.

Tras finalizar el montaje y el sistema del OpenROV, aprendí a solventar imprevistos que no se contemplaban, tanto de hardware como de software. En este aspecto, la comunidad de OpenROV ha sido de inestimable ayuda en algunos momentos.

Se comprobó que el robot es estanco y puede navegar bajo la superficie marina, que todos los controles mencionados en los puntos anteriores funcionan correctamente (láser, cámara, luces, etc) y que pueden aparecer algunos de los problemas comunes (por ejemplo, el vaho).

Debido a que OpenROV tiene software libre, se pudo instalar los paquetes necesarios para realizar la comunicación con ROS y se comprobó que existía la comunicación entre las dos plataformas. 

Se podrían mejorar ciertos aspectos del OpenROV si se siguen implementando nodos de ROS para su utlilización. Por ejemplo, para la cámara podría usarse el paquete sensors; para la velocidad podría utilizarse geometry\_msgs; etc. 

Para finalizar, otro de los siguientes hitos que se podrían realizar con este TFG es intentar descubrir si la leyenda de Nessie es cierta.
